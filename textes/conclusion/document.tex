Les programmes de recherche sur \enquote{La couleur~: artefacts, matière et cognition} et sur \enquote{La fabrique matérielle du visuel~: panneaux peints en Méditerranée} ont abouti à trois projets de \indexmot{visualisation}~: une \indexmot{carte interactive}, une \indexmot{frise chronologique} et une édition de fac-similés numériques. Ces \indexmot{visualisation}s prolongent et enrichissent les recherches des années précédentes. Elles permettent de réexaminer les données collectées et les résultats obtenus, elles facilitent la communication et la médiation de la recherche, et elles instaurent un langage commun entre diverses cultures scientifiques. Ainsi, une politique de \indexmot{visualisation} des données doit être intégrée dans la réflexion du chercheur dès le lancement d’un programme de recherche.\\\par
Ce mémoire a démontré que les \indexmot{visualisation}s ne sont pas simplement des outils à disposition du chercheur, mais qu'elles produisent et reflètent également des connaissances historiques. La cartographie interactive et la \indexmot{frise chronologique} proposent ainsi une mise en ordre du temps et de l’espace selon l’histoire des enluminures des manuscrits et des panneaux peints. Chaque réalisation est située par rapport aux autres, en fonction d’un système de repérage constitué de coordonnées ou de dates. Il s’agit de représenter, de distinguer ou de regrouper une réalisation par rapport à une autre, selon leur contexte géographique ou temporel.\par
Ces \indexmot{visualisation}s sont à l’origine d’un discours sur l’espace et sur le temps pour le chercheur. Elles permettent d’identifier des tendances historiques et des contextes de production, autrement dit des alignements de données, qui doivent être explorés. Le chercheur, parfois submergé par les milliers de lignes d’un fichier~CSV, peut perdre de vue l’ensemble des données. Les \indexmot{visualisation}s redonnent un sens aux analyses produites, les orientant selon des préoccupations chronologiques et géographiques. Elles permettent également de faire ressortir les données essentielles et de reléguer les autres à leur place accessoire ou complémentaire, grâce à des politiques d’éditorialisation de la base, qui, comme nous l’avons vu, est avant tout un encodage de filtres sur les \indexmot{thésaurus} constitués.\par
Les fac-similés numériques, troisième et dernier volet des \indexmot{visualisation}s, facilitent la communication et la médiation de la recherche. Les données de la base sont reportées sur des images numériques en haute résolution avec des métadonnées de présentation. Les analyses menées, qui constituent le travail scientifique des chercheurs lors du projet de recherche, deviennent intelligibles au public parce que contextualisées sur la source même du travail. Le fac-similé numérique permet aussi d’instaurer un langage commun entre des cultures scientifiques diverses. Ainsi, le scientifique et les historiens, historiens de l’art, spécialistes des textes anciens ont sous les yeux le document qui est l’objet de leur discussion. Ils peuvent y rapporter directement leurs commentaires, leurs analyses et leurs résultats. Des étapes de traduction scientifique, qui peuvent parfois être à la source d’erreurs, n'ont plus lieu d’être et les processus de communication sont simplifiés.\\\par
Ce mémoire fait apparaître que la \indexmot{visualisation} n’est plus seulement la représentation de l’analyse d’un chercheur avec des outils informatiques. Dans un certain sens, une logique s’inverse~: la \indexmot{visualisation} n’est plus une fin, mais tend à devenir un moyen, une étape de la recherche. Avec cette redéfinition, le public de la \indexmot{visualisation} change aussi. Elle ne se réduit plus au grand public, à des fins de communication de la recherche, mais elle devient un service de la recherche, désormais en soutien aux chercheurs.\par
Cette évolution conduit à repenser les différentes séquences des projets de recherche. D’un schéma en trois temps, où se succèdent \enquote{recherche}, \enquote{analyse} et \enquote{publication (dont \indexmot{visualisation})}~; il conviendrait de passer à une nouvelle articulation \enquote{recherche}, \enquote{traitement informatique (dont \indexmot{visualisation}) et analyse}, puis \enquote{publication}. La réunion de l’analyse et du traitement informatique est déjà une réalité dans les projets de recherche actuels et ne cesse de se renforcer, en témoigne la place croissante prise par les ingénieurs d’étude et de recherche à côté des chercheurs dans le monde de la recherche.