\texttt{Leaflet}\par
La carte interactive proposée avec la bibliothèque JavaScript Leaflet fonctionne sans aucun défaut. Toutes ses fonctionnalités sont opérationnelles, qu'il s'agisse de l'affichage de la carte, des différents marqueurs et de leurs informations dans le carrousel et la barre latérale, ou de la combinaison des différents filtres. Cependant, je crois qu'une option supplémentaire aurait pu améliorer encore l'expérience de navigation et d'exploration des résultats. Actuellement, la carte interactive dévoile les résultats par feuillet, alors que les filtres portent sur les analyses. Ainsi, si le résultat d’un filtre ne correspond qu’à la troisième analyse d’un feuillet, il est nécessaire de faire défiler le carrousel jusqu’à l’apparition de la bonne analyse. Cet inconvénient ralentit quelque peu l’exploration de la carte, bien que les informations affichées demeurent correctes.\par
J'aurais souhaité, si la durée du stage me l’avait permis, de réaliser une seconde bascule dans les données, semblable à celle proposée pour le projet. Cette fois-ci, l’utilisateur aurait eu le choix de consulter les résultats par feuillet ou par manuscrit. Le carrousel aurait alors défilé les différentes pages analysées du manuscrit correspondant. Cette option aurait permis à l’utilisateur d’avoir une idée plus précise du nombre de manuscrits répondant à ses critères de sélection, ce qui est peut-être plus pertinent que d’obtenir le résultat par feuillet analysé.\\\par
\texttt{VIKUS Viewer}\par
La frise chronologique développée avec VIKUS Viewer présente un défaut majeur lors de son utilisation. Bien que l’articulation entre les différents paramètres fonctionne correctement et que les images apparaissent comme prévu lors de l'application des filtres, la cohérence de la barre latérale avec le feuillet affiché peut poser problème après un certain temps d’utilisation. Après une période indéterminée de navigation, les informations contenues dans la barre latérale se figent et ne correspondent plus à celles du feuillet sélectionné. Il est alors nécessaire de rafraîchir la visualisation. Ce dysfonctionnement est problématique pour les utilisateurs : non seulement ils doivent à nouveau renseigner leurs paramètres de recherche, mais ils doivent également faire preuve d'une grande vigilance pour éviter de recopier des informations erronées. J'ai inclus un avertissement dans le guide d’utilisation de la visionneuse à ce sujet. Cependant, ce défaut persiste et pose un réel problème. Il me semble qu'il pourrait être dû à la manière dont VIKUS déclare les variables et les met à jour. Malheureusement, dans le cadre de mon stage de trois mois, je n'ai pas eu l'occasion de pousser plus loin l’investigation ni de proposer une correction aux développeurs.\\\par
\texttt{Annotations IIIF}\par
Les annotations présentes dans les manifestes IIIF ne correspondent pas à celles initialement prévues. Le serveur d’annotations SAS permet de sélectionner un type de marqueur qui me semble bien plus précis que les carrés retenus et qui seront déployés sur le site Omeka~S. Après une phase de tests pour quelques annotations, j’ai constaté que les données~SVG (Scalable Vector Graphics) qui définissent ce marqueur étaient trop complexes pour la version de Mirador prévue pour Omeka~S. J’ai donc dû simplifier ces données et opter pour la représentation des analyses par petits carrés. De même, la version~3 de Mirador affiche le contenu des annotations dans une barre latérale, contrairement à la version précédente où elles apparaissaient dans une info-bulle. Cette modification rend l’information moins lisible.\par
De plus, l’implémentation de la visionneuse dans Omeka~S pose un problème quant à la gestion de certaines fonctionnalités de IIIF. Comme déjà souligné au cours de ce mémoire, il m’était impossible de proposer un affichage par calques, tel que je l’aurais souhaité initialement. Tous les calques se superposaient, et la seule solution a été d’abandonner cette possibilité. En revanche, la migration des données a été satisfaisante et toutes les informations qui devaient apparaître y figurent bien.