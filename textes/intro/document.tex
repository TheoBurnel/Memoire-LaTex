\section{L’exemple du \indexmot{bleu égyptien}, ou ce que l’investigation physico-chimique apporte à l’histoire de l’art}

Les couleurs connaissent des trajectoires, changeantes avec les époques, les lieux et les découvertes. Elles témoignent des connaissances d'une société, des modes d'un milieu, des besoins, aussi, des artistes et des commanditaires. Les couleurs sont un reflet des savoirs techniques de leur création, des échanges commerciaux pour leur acheminement et d’une interprétation sémantique de leur usage, de la force symbolique de leur choix. La \enquote{fabrique visuelle} d’une enluminure n'est pas qu'une création artistique, à moins de considérer cette dernière comme le miroir de ce qui précède \footcite[Pour une introduction à la trajectoire des couleurs travers le temps, voir][]{pastoureau_petit_2014}.\par
L’histoire de l'utilisation du bleu d’Alexandrie, communément appelé \enquote{\indexmot{bleu égyptien}}, semble être un parfait exemple de ce que la présence – ou l'absence – d'une couleur peut nous apprendre sur la localisation et la datation des œuvres, sur les pratiques d’un artiste ou d’un atelier, sur les différentes interventions au sein d’une même œuvre, sur les circuits commerciaux et la circulation des pigments ou encore sur les transferts techniques entre les cultures.\par
Le bleu ne se résume pas à une couleur, mais à une multitude de teintes et de nuances, tantôt lumineuses, tantôt sombres. Parmi les variétés de bleu, l’égyptien se démarque par son origine : il est le plus ancien pigment de synthèse connu. Charlotte Denoël, dans l’édition critique du \textit{\indexmot{Beatus de Saint-Sever}}, retrace son histoire~:\par
\begin{quote}
« Apparu en Égypte sous la IVe dynastie, vers 2620 av. J.-C., et très employé en peinture, notamment sur les portraits du Fayoum, ce composé de silicate double de cuivre et de calcium était obtenu artificiellement par cuisson à 850°C dans des conditions très contrôlées, avec un savoir-faire élaboré, le plus souvent au contact de métallurgistes. C’est pourquoi ce pigment est longtemps passé pour disparu avec la fin du monde antique, sa toute dernière occurence apparaissant dans le célèbre Pentateuque d’Ashburnham exécuté aux environs de 600, probablement à Rome. De récentes analyses ont cependant permis d’attester la présence du \indexmot{bleu égyptien} dans quelques peintures murales et manuscrits enluminés du premier Moyen Âge, dont l’Évangéliaire de Charlemagne, daté de 781-783. Son identification dans le Beatus vient documenter de façon remarquable la permanence de l’utilisation de ce pigment au XIe siècle, en même temps qu’elle souligne l’extraordinaire niveau de technicité des artistes de Saint-Sever.\footnote{\cite{denoel_beatus_2022}. Pour son utilisation dans l’évangéliaire de Charlemagne, voir~: \cite{roger_etude_2007}. Pour une présentation des l’analyse physico-chimique du bleu : \cite{gameson_pigments_2023}.} » 
\end{quote} \par
Des investigations physico-chimiques permettent ainsi de retracer et de comprendre des conditions matérielles et un contexte iconographique d’utilisation de ce pigment. Le \indexmot{bleu égyptien} s’inscrit dans une \indexmot{chronologie} et une \indexmot{géographie} qui, par l’analyse d’un corpus de manuscrits, s’éclaircissent progressivement. Il en est du \indexmot{bleu égyptien} comme d’autres couleurs et le projet de recherche sur la \enquote{La couleur~: artefacts, matière et cognition}, qui a obtenu un financement dans le cadre de l’appel 2020-23, permet de proposer aux chercheurs en sciences des matériaux du patrimoine, aux historiens ou aux historiens de l’art des informations précieuses sur les enluminures des manuscrits.

\section{Une recherche scientifique et culturelle}

Le cas du \indexmot{bleu égyptien} semble représentatif des nouvelles recherches qui sont menées sur les enluminures médiévales depuis une dizaine d’années maintenant. Plus que d’autres recherches, les travaux sur les manuscrits anciens lient, avec bénéfice, les différents apports de l’interdisciplinarité – qui dépassent ceux des sciences humaines. La recherche sur la couleur, la matière et la technique utilisées dans la composition des enluminures permet d’enrichir scientifiquement et culturellement nos connaissances sur ce monde passé et encore préservé, où chefs-d’œuvre textuels et picturaux cohabitent.\footnote{La création en 2022 d’un musée à la \indexmot{Bibliothèque nationale de France} sur le site de Richelieu participe à la connaissance du grand public des trésors de la bibliothèque, dont les manuscrits occupent la première place.}. \par
La découverte est dans un premier temps scientifique. Elle s’appuie sur les dernières techniques mises au point dans l’investigation des manuscrits – des techniques qui, à la différence d’un passé récent encore, sont désormais non invasives et permettent d’explorer les manuscrits les plus précieux et en plus grand nombre. Il n’est pas lieu ici de présenter les différentes techniques qui permettent la réalisation des analyses physico-chimiques~; les personnes intéressées pourront se référer au chapitre dédié dans l’ouvrage \textit{The pigments of british medieval illuminators}\footcite[Pp. 1-42.][]{gameson_pigments_2023}. À titre d’exemple, dans le cas des bleus rencontrés dans le \textit{Beatus de Saint-Server}, trois technologies sont combinées afin de produire une analyse ponctuelle, une cartographie de la surface des feuillets et une de ses différentes couches~:\par 
\begin{quote}
	« […] la tomographie en cohérence optique, pour rendre compte de la structure des couches colorées et de leurs épaisseurs~; l’imagerie hyperspectrale dans le visible et dans l’infrarouge, pour accéder à la réflectance des matériaux et identifier certaines espèces colorantes~; la cartographie de fluorescence X et UV pour rendre compte de l’analyse élémentaire composant les matériaux et renseigner certains composés de notre organique ou inorganique \footcite[Pp. 78.][Ces analyses ont été réalisées par le centre de recherche et de restauration des musées de France et le Centre de recherche sur la conservation.]{denoel_beatus_2022} ». \par
\end{quote}
Ces analyses permettent de mettre en avant différentes teintes, nuances et techniques autour de la couleur bleu. Dans le seul cas de la représentation du paon et du serpent dans le \textit{Beatus de Saint-Server}, quatre bleus sont ainsi identifiés~: l’indigo, l’azurite, l’outremer et, enfin, le \indexmot{bleu égyptien}\footnote{Cf. annexe 1.}. La découverte devient aussi culturelle, tant l’analyse apporte des éléments sur nos connaissances des colorants, des matériaux, des enlumineurs et des manuscrits d’une époque. L’exploration d’un corpus aussi important que celui réalisé dans le cadre du projet \enquote{La couleur~: artefacts, matière et cognition} permet d’interroger à nouveau frais la circulation des matières premières essentielles à l’enlumination, la sémantique des couleurs ou encore la localisation et la temporalité des productions artistiques. \par
Les analogies, les rapprochements sont essentiels à l’heuristique dans le cadre d’une histoire transculturelle sur plus d’un millénaire et marquée par une pauvreté des sources primaires comme il en est ici. Mais, sans rien ôter à la singularité de chaque manuscrit, en témoigne le regard porté sur les artistes eux-mêmes, la comparaison des résultats met en avant des transferts ou des influences culturelles qui peuvent représenter de nouvelles interprétations et analyses à la réalisation des enluminures sur un temps long. Les questions sont nombreuses et quelques exemples peuvent illustrer de la richesse de la recherche : comment apparaît et se diffuse l’utilisation du vert de cuivre? le sens d’un rouge ou d’un pourpre dans une enluminure change-t-il au cours du temps? le jaune est-il nécessairement doré pour le sacré? Les réponses à ces différentes interrogations trouvent leur élucidations dans la comparaison des occurrences du \indexmot{thésaurus} constitué et dans son architecture\footnote{Je reviendrai plus longuement sur l’importance du \indexmot{thésaurus} dans la première partie de ce mémoire.}. Un répertoire structuré de termes sur des critères comme les matières colorantes, les objets, les supports, les centre de production, les époques et les aires géographiques est essentiel au comparatisme. Les recherches conduites dans le cadre du projet, sur près de cinq milles couches matérielles, permettent dès lors l’élaboration d’un corpus quantitatif, qui vient s’ajouter à l’existence des données qualitatives obtenues par les analyses. Ces données, complexes et disséminées dans l’immensité de la recherche effectuée, peuvent être utilisées, avec bénéfice, dans le cadre d’une projet de \indexmot{visualisation}.

\section{La \indexmot{visualisation}~: la représentation au service de l’interprétation}

Les apports de la conduite d’un projet de \indexmot{visualisation} pour accompagner la recherche scientifique sont aujourd’hui suffisamment reconnus pour qu’il ne faille pas à nouveau les expliciter et les justifier ici. Mais plus que des apports, ces projets sont en passe de devenir des conditions \textit{sine qua non} à l’obtention de contrats de recherche. La \indexmot{visualisation}, définie comme une représentation de données, est conçue pour aider à comprendre, interpréter et communiquer des informations et des résultats de manière efficace. La \indexmot{visualisation} n’est pas seulement la dernière séquence d’un projet scientifique qui, d’une certaine façon, matérialiserait la recherche menée – un faire-valoir des résultats~; elle est essentielle à tous les niveaux de la recherche, depuis la collecte et l’analyse des données jusqu'à la communication des résultats. Dans le cadre d’un projet comme celui qui m’intéresse dans le cas présent, il me semble que la conduite d’un projet de \indexmot{visualisation} comporte au moins trois avantages.
\begin{enumerate}
	\item Elle participe à la compréhension des données. Les \indexmot{visualisation}s permettent aux chercheurs d’interpréter rapidement et intuitivement des données initialement complexes, parfois illisibles – illisibles dans le sens où elles nécessitent de parcourir et de retenir près de cinq mille lignes pour en comprendre le sens. En représentant des relations entre des variables, comme par exemple des années, des couleurs ou des matériaux, des tendances et des modèles peuvent être identifiés plus facilement. Les analyses peuvent ainsi être multipliées en des temps réduits. Les \indexmot{visualisation}s permettent aux chercheurs de repérer d’autant plus facilement des tendances lorsqu’une option d’éditorialisation des jeux de données est mise en place\footnote{Les chercheurs ont la possibilité d’éditorialiser leur recherche sur la \indexmot{carte interactive} et la \indexmot{frise chronologique}.}. Les chercheurs peuvent alors explorer les données sous différents angles et identifier des associations qui pourraient ne pas être évidentes autrement.
	\item La communication et la méditation des résultats sont facilitées. Les représentations rendent les résultats de la recherche plus accessibles et attrayants pour un public plus large, y compris les chercheurs dans d'autres domaines. L’aridité de la recherche scientifique, l’initiation préliminaire qu’elle demande, sont compensées par une clarté et une simplicité inhérente à la \indexmot{visualisation}. L’exemple de la \indexmot{carte interactive} qui sera présenté plus avant me semble parfaitement éclairant sur ce point~: la concentration ou, au contraire, la dissémination des marqueurs, qui représentent les résultats du filtre dans les jeux de données, est parfaitement interprétable par un public qui ne serait pas familiarisé avec les manuscrits et les enluminures.
	\item Elle favorise la collaboration interdisciplinaire. Les \indexmot{visualisation}s peuvent servir de point de départ pour des discussions collaboratives entre chercheurs de différents domaines. En représentant visuellement les données, les chercheurs peuvent mieux comprendre et apprécier les contributions des autres membres de l'équipe, ce qui favorise la collaboration interdisciplinaire. Dans le cadre d’annotations sémantiques d’images à partir des technologies \indexmot{IIIF}, il me semble ainsi que le dialogue entre l’historien de l’art, le conservateur et le scientifique réalisant l’analyse physico-chimique est renforcé et simplifié par la désignation précise de l’attendu et sa représentation.
\end{enumerate} \par
En somme, la conduite d’un projet de \indexmot{visualisation} me semble tenir un rôle primordial lors d’un projet de recherche. Elle permet d’interroger à nouveau frais les données récoltées et les résultats obtenus, elle facilite la communication et la médiation de la recherche, et elle peut instaurer un langage commun entre les cultures scientifiques diverses. Dès lors, la représentation des données ne saurait être reléguée au second plan, marginalisée ou conditionnée à un reliquat budgétaire. Elle est et doit être présente dans la réflexion du chercheur, au même titre que celle sur la communication écrite, qu’elle soit d’articles scientifiques ou de vulgarisation. \newline\par
Afin d’appuyer cette hypothèse, ce mémoire est composé en quatre temps. Le premier chapitre propose un retour sur les préliminaires de la conduite d’un projet de \indexmot{visualisation}. Il s’agit d’un chapitre introductif qui revient sur l’antécédent de la base de données qui sert de socle à la \indexmot{visualisation}. Le deuxième et le troisième chapitres retracent les choix et la réalisation de l’éditorialisation des données de la base. Ils présentent les réflexions et les étapes de la construction d’une \indexmot{carte interactive} et d’une \indexmot{frise chronologique}, les deux représentations qui permettent d’ordonner les données selon un espace et un temps – en d’autres termes, de rendre intelligible un fichier CSV à plusieurs milliers de lignes. Ce mémoire se clôture par la présentation de la réalisation de fac-similés numériques. L’édition enrichie des manuscrits selon ce modèle permet une nouvelle communication de la recherche auprès du public, plus interactive et attractive.